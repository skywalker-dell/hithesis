

\chapter[相机选型与基本设置]{相机选型与基本设置}[Harbin Institute of Technology Postgraduate Dissertation Writing Specifications]

\section{引言}[Content specification]
相机是计算机视觉中非常重要的一部分,负责获取环境中的图像信息,为后续的图像处理和算法提供输入数据。
因此,在进行计算机视觉项目时,选择适合的相机和合理的设置相机参数设置是非常关键的。
本章根据RoboMaster比赛的赛场光照环境和检测算法要求选择合适的工业相机并给出具体的型号。

\section{相机选型策略}[Content specification]
尽量选择高帧率、感光片大的相机,在此基础上选择光圈大的镜头,焦距选择视情况而定。
下面详细阐述:\par

\begin{itemize}[itemindent=2em]
    \item 分辨率:一般分辨率越高,图像质量越好,越有利于识别算法和测距算法。但分辨率越高,图像最大采集帧率越低,
    且检测算法耗时增加,价格也相应越高。因此选择分辨率在30万到100万像素的相机即可。
    \item 帧率:帧率越高,运动图像捕捉能力越好。由于我们需要检测高速移动或者旋转的目标,因此帧率越高越好。
    根据应用场景需要捕捉的运动速度和动态要求,选择适合的帧率,对于RoboMaster比赛场景来说,150fps+的帧率能够满足基本要求。
    \item 传感器尺寸和类型:CMOS和CCD是现代数字图像传感器的两种常见类型。
    CMOS传感器具有低功耗、低噪声、高帧率等优点,且质量更小,所以选择CMOS相机。
    \item CMOS传感器有两种快门方式,卷帘快门和全局快门。
    全局快门曝光时间更短,但噪声更大;卷帘快门可以达到更高的帧速,但当
    曝光时间较长或物体移动较快时产生果冻效应。
    为了适应RoboMaster比赛场景的需求,选择全局快门的相机。
    \item 工作温度和防护等级:根据应用环境和需求,选择适合的工作温度和防护等级。
    工业相机一般需要在恶劣环境下工作,如高温、灰尘等,所以防护等级要求较高。
    \item 软件支持:算法运行环境为Linux,
    工业相机的软件平台需要支持图像处理和数据分析等功能,
    通常需要与特定的软件开发工具包配合使用,且售后的技术支持和SDK软件包的是否适合二次开发也是选择的考量重点。

\end{itemize}


一张成像清晰、亮度均匀的图片非常有利于后续的图像处理算法,因此如何设置相机参数,以及如何预处理原始图像后都非常重要。
\section{相机硬件设置}[Content specification]
拿到相机后,基本设置如下:将光圈拧到最大;调整成像平面与光心的距离,使得5m成像最清晰;涂抹螺丝胶,防止相机在剧烈震动的情况下因螺丝松动而成像不清。之所以将光圈拧到最大,是为了在达到相同图像亮度的情况下获得更短的曝光时间,进而提高相机帧率。
\section{相机软件设置}[Content specification]
模拟增益调高,目的:一是在达到相同图像亮度的情况尽可能减少曝光时间从而提高取图帧率,
不过由于增益拉高后图像噪点也会增加,因此需要平衡; 
二是使得发光体在图像中更加明显,从而二值化阈值可以取到非常高,可以防止过曝给灯条的提取造成影响,
因为即使在过曝的情况下,光晕的灰度值与真正的发光体还是有较大的区别,通过高阈值筛选可以很好的提取灯条。

设置相机白平衡参数,使得成像色彩正常,
迈德威视相机的软件工具包提供了自动调节白平衡参数的功能,只需要将相机对准白色物体,
然后点击“白平衡”按钮即可。

\section{本章小结}[Content specification]
本章主要介绍了相机选型、硬件设置和软件设置等方面的知识。
在相机选型方面,需要考虑分辨率、帧率、传感器尺寸和类型、工作温度和防护等级、
以及软件支持等因素。在硬件设置方面,需要注意光圈的设置和成像平面与光心的距离等。
在软件设置方面,需要注意模拟增益的调整和颜色通道增益的选择等。
明确这些方面的知识,可以使相机的使用更加得心应手,也能够为后续的图像处理和算法提供更好的输入数据。
\par
最终选择迈德威视相机,型号为MV-SUA34GC。规格参数在附录中。
% \begin{table}[htbp]
%     \centering
% \begin{tabular}{ |p{3cm}||p{3cm}|p{3cm}|p{3cm}|  }
%     \hline
%     \multicolumn{4}{|c|}{Country List} \\
%     \hline
%     Country Name     or Area Name& ISO ALPHA 2 Code &ISO ALPHA 3 Code&ISO numeric Code\\
%     \hline
%     Afghanistan   & AF    &AFG&   004\\
%     Aland Islands&   AX  & ALA   &248\\
%     Albania &AL & ALB&  008\\
%     Algeria    &DZ & DZA&  012\\
%     American Samoa&   AS  & ASM&016\\
%     Andorra& AD  & AND   &020\\
%     Angola& AO  & AGO&024\\
%     \hline
%    \end{tabular}
%    \caption{MV-SUA34GC}
%    \label{tab:example}
% \end{table}