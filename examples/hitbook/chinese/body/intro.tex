% !Mode:: "TeX:UTF-8"

\chapter[绪论]{绪论}[Harbin Institute of Technology Postgraduate Dissertation Writing Specifications]

\section{课题背景及研究目的和意义}[Content specification]

本研究基于全国大学生机器人大赛RoboMaster,
旨在为战队研发一种高效鲁棒的自动瞄准算法,
实现对装甲板的识别、坐标解算、运动建模、预测预测等算法。
一键瞄准、自动锁定,帮助操作手提高装甲板的命中率,从而取得比赛的胜利。
同时,该研究涉及大量的机器人领域的知识,并涉及实际的工程开发,学以致用,提高学生对专业领域知识的掌握程度。

\section{研究现状}

\subsection{目标检测算法研究现状}
目标检测是计算机视觉领域中的一项重要任务,旨在从图像或视频中检测出目标物体的位置和类别。
目标检测算法的研究一直是计算机视觉领域的热点之一。
目前,主要的目标检测算法可以分为两类:基于传统机器学习方法的算法和基于深度学习方法的算法。\par


传统机器学习方法主要包括特征提取和分类两个步骤。
常用的特征提取算法有Haar、HOG和SIFT等。
其中,Haar特征和HOG特征主要用于行人检测,SIFT特征主要用于物体检测。
分类算法主要包括Adaboost、SVM和Random Forest等。
这些算法在一定程度上可以实现目标检测,但是它们的检测效果和速度相对较低,已经逐渐被深度学习算法所替代。

\par

随着深度学习算法的发展,基于深度学习的目标检测算法逐渐成为主流。
常用的深度学习模型包括Faster R-CNN、YOLO、SSD、RetinaNet和Mask R-CNN等。
这些算法在目标检测准确率和速度上都有很大的提升,已经广泛应用于各个领域,例如自动驾驶、安防监控、医疗诊断等。
\par
此外,当前目标检测算法研究的重点主要集中在以下几个方面:
\begin{itemize}[itemindent=2em]
    \item 目标检测算法的精度和速度的平衡问题,尤其是在实时应用中需要同时考虑两者的性能指标。
    
    \item 目标检测算法的可解释性问题,即如何让算法输出的检测结果更容易被人理解和解释。
    
    \item 小样本目标检测,即在少量样本的情况下实现准确的目标检测,这对于一些特定领域的应用非常重要。
    
    \item 目标检测算法在复杂场景下的应用,例如低光照、遮挡、姿态变化等情况下的检测效果如何提升。
    
    \item 错误提示:clangd可以帮助用户发现代码错误,并给出相应的提示,方便用户快速定位和修复代码问题。
\end{itemize}


总之,目标检测算法的研究还有很大的发展空间,未来的研究方向将更加注重模型的实用性和鲁棒性,以满足实际应用的需求。

\subsection{卡尔曼滤波器研究现状}
卡尔曼滤波器是一种经典的状态估计方法,可用于估计时间序列的未知状态。在目标跟踪、自动导航和机器人等领域中得到广泛应用。目前,卡尔曼滤波器的研究现状主要包括以下几个方面:
\par
卡尔曼滤波器的基础理论。
卡尔曼滤波器的基础理论已经相对成熟,包括线性系统、高斯噪声和线性状态估计等方面的理论研究,这些理论为卡尔曼滤波器的应用提供了坚实的基础。
\par
卡尔曼滤波器的改进和优化。
为了提高卡尔曼滤波器的估计精度和稳定性,一些改进和优化方法被提出,如扩展卡尔曼滤波器、无迹卡尔曼滤波器、粒子滤波器等。这些方法使卡尔曼滤波器能够更好地适应非线性系统和非高斯噪声的情况,并取得了一定的效果。

\subsection{多目标追踪研究现状}
多目标追踪预测算法是指在视频或图像序列中,对多个目标的轨迹进行跟踪和预测的算法。
目前,多目标追踪预测算法的研究主要集中在以下几个方面:
\par
基于传统方法的算法。
传统的多目标追踪算法主要包括卡尔曼滤波器、粒子滤波器、条件随机场和蒙特卡罗方法等。这些算法在目标跟踪和预测中有着广泛的应用,但是在处理复杂场景和高速移动目标时,精度和效率存在一定的限制。
\par
基于深度学习方法的算法。
深度学习方法在多目标追踪预测算法中也有广泛的应用。主要的深度学习模型包括Faster R-CNN、Mask R-CNN、YOLO、SORT、Deep SORT等。这些算法主要利用深度学习网络进行目标检测和特征提取,并使用各种跟踪方法进行目标跟踪和预测。这些算法的优势在于准确率高、鲁棒性好,但是计算量大,处理速度相对较慢。
\par
结合传统方法和深度学习方法的算法。
为了充分利用传统方法和深度学习方法的优势,一些研究者将两种方法进行结合,提出了一些新的算法。例如,将卡尔曼滤波器和深度学习方法相结合,可以提高算法的鲁棒性和实时性;将多个深度学习模型进行融合,可以提高算法的准确率。


\section{本文的主要研究内容}
本文的研究内容包括以下几点:
\begin{itemize}[itemindent=2em]
    \item 适用于RoboMaster比赛环境的工业相机选型。
    \item 基于通信的主从机时钟对齐。
    \item 将传统数字图像处理技术与深度学习相结合的目标检测算法。
    \item 基于卡尔曼滤波器的运动物体建模、预测算法。
    \item 基于物体运动模型的多目标追踪算法。
\end{itemize}





