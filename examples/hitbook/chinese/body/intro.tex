% !Mode:: "TeX:UTF-8"

\chapter[绪论]{绪论}[Harbin Institute of Technology Postgraduate Dissertation Writing Specifications]

\section{课题背景及研究目的和意义}[Content specification]

本研究基于全国大学生机器人大赛RoboMaster,
旨在为战队研发一种高效鲁棒的自动瞄准算法,
实现对装甲板的识别、坐标解算、运动建模、预测等算法。
一键瞄准、自动锁定,帮助操作手提高弹丸击打装甲板的命中率,从而取得比赛的胜利。
同时,该研究涉及大量的机器人领域的理论知识,
并涉及实际的工程开发。学以致用,提高对专业领域知识的掌握程度。

\section{研究现状}

\subsection{目标检测算法研究现状}
目标检测是计算机视觉领域中的一项重要任务,旨在从图像或视频中检测出目标物体的位置和类别。
目标检测算法的研究一直是计算机视觉领域的热点之一。
目前,主要的目标检测算法可以分为两类:基于传统机器学习方法的算法和基于深度学习方法的算法。\par


传统机器学习方法主要包括特征提取和分类两个步骤。
常用的特征提取算法有方向梯度直方图(Histogram of Oriented Gradient, HOG)特征\cite{pang2011efficient}和尺度不变特征转换(Scale-invariant feature transform, SIFT)特征\cite{ng2003sift}等。
其中,HOG特征主要用于行人检测,SIFT特征主要用于物体检测。
分类算法主要包括自适应增强(Adaptive Boosting, Adaboost)\cite{hastie2009multi}、支持向量机(Support Vector Machine, SVM)\cite{schuldt2004recognizing}和随机森林(Random Forest, RF)\cite{biau2016random}等。
这些算法在一定程度上可以实现目标检测,但是它们的检测效果和速度相对较低,已经逐渐被深度学习算法所替代。

\par

随着深度学习算法的发展,基于深度学习的目标检测算法逐渐成为主流。
常用的深度学习模型包括快速区域卷积神经网络(Fast R-CNN)\cite{girshick2015fast}、基于单次检测的目标检测算法(You Only Look Once, YOLO)\cite{jiang2022review}等。
这些算法在目标检测准确率和速度上都有很大的提升,已经广泛应用于各个领域,例如自动驾驶、安防监控、医疗诊断等。
\par
此外,当前目标检测算法研究的重点主要集中在以下几个方面:
\begin{itemize}[itemindent=2em]
    \item 目标检测算法的精度和速度的平衡问题,通常称为精度-速度权衡(trade-off)。
    这是因为在目标检测任务中,算法需要同时实现高精度的目标定位和分类,同时保持较快的处理速度。
    在追求更高精度的同时,算法可能需要更复杂的模型和更多的计算资源。
    这可能导致算法的处理速度变慢,不适用于实时应用或大规模数据集的处理。
    相反,追求更快的处理速度可能会牺牲一定的精度,导致错误的目标检测或较低的定位精度。
    
    \item 目标检测算法的可解释性问题,即如何让算法输出的检测结果更容易被人理解和解释。
    目标检测算法的可解释性问题指的是对算法内部决策和输出结果的解释和理解能力。
    在某些应用场景中,仅仅依靠算法的高准确率可能是不够的,还需要能够理解算法是如何做出决策的、
    哪些特征导致了目标的检测等。这对于一些对决策过程有强解释需求的应用领域尤其重要,
    比如医疗诊断、自动驾驶等。
    
    \item 小样本目标检测,即在少量样本的情况下实现准确的目标检测,这对于一些特定领域的应用非常重要。以下是关于小样本目标检测的一些研究热点:
    迁移学习和预训练模型\cite{zhuang2020comprehensive}:迁移学习使用在大规模数据集上预训练的模型作为初始权重,并通过微调或其他适应性方法来调整模型以适应小样本目标检测任务。预训练模型,可以提供在大规模数据上学习的通用特征,从而加快在小样本数据上的学习。
    元学习\cite{hospedales2021meta}:元学习(或称为学习到学习)是一种可以在小样本情况下进行目标检测的方法。通过在大量不同任务上进行训练,元学习模型可以学习到一种学习策略,可以快速适应新任务。
    生成对抗网络(Generative Adversary Network, GAN):生成对抗网络可以通过生成逼真的合成样本来增强小样本目标检测任务的训练。
    具有注意力机制的模型:注意力机制可以帮助模型关注关键区域,从而在小样本目标检测任务中提升性能。
    
    \item 目标检测算法在复杂场景下的应用,例如低光照、遮挡、姿态变化等情况下的检测效果如何提升。
    
\end{itemize}


总之,目标检测算法的研究还有很大的发展空间,未来的研究方向将更加注重模型的实用性和鲁棒性,以满足实际应用的需求。

\subsection{卡尔曼滤波器研究现状}
卡尔曼滤波器是一种经典的状态估计方法,可用于估计时间序列的未知状态。在目标跟踪、自动导航和机器人等领域中得到广泛应用。目前,卡尔曼滤波器的研究现状主要包括以下几个方面:
\par
卡尔曼滤波器的基础理论。
卡尔曼滤波器的基础理论已经相对成熟,包括线性系统、高斯噪声和线性状态估计等方面的理论研究,
这些理论为卡尔曼滤波器的应用提供了坚实的基础。
\par
卡尔曼滤波器的改进和优化。
为了提高卡尔曼滤波器的估计精度和稳定性,一些改进和优化方法被提出:
扩展卡尔曼滤波器(Extended Kalman Filter, EKF)\cite{ribeiro2004kalman}是对非线性系统进行状态估计的扩展。它通过在非线性函数上进行泰勒级数展开,将非线性系统近似为线性系统,从而应用线性卡尔曼滤波器的方法。扩展卡尔曼滤波器在非线性系统的状态估计中得到广泛应用,并且已经成为卡尔曼滤波器的标准扩展之一。
无迹卡尔曼滤波器(Unscented Kalman Filter, UKF)\cite{wan2000unscented}是对扩展卡尔曼滤波器的改进,旨在提高非线性系统的状态估计精度。它通过选择一组特定的采样点(称为无迹变换)来近似非线性系统的均值和协方差。无迹卡尔曼滤波器相比于扩展卡尔曼滤波器在非线性系统估计方面具有更好的稳定性和准确性。
粒子滤波器(Particle Filter)\cite{gustafsson2010particle}基于贝叶斯滤波理论,通过使用一组粒子来表示状态的后验概率分布。每个粒子都代表了系统状态的一个假设或样本。粒子的权重表示了该假设的后验概率。粒子滤波器通过递归地更新粒子的权重和位置,来逼近并估计系统的后验概率分布。
这些方法使卡尔曼滤波器能够更好地适应非线性系统和非高斯噪声的情况,
并取得了一定的效果。

\subsection{多目标追踪研究现状}
多目标追踪预测算法是指在视频或图像序列中,对多个目标的轨迹进行跟踪和预测的算法。
目前,多目标追踪预测算法的研究主要集中在以下几个方面:
\par
基于传统方法的算法。
传统的多目标追踪算法主要包括卡尔曼滤波器、粒子滤波器、条件随机场和蒙特卡罗方法等。
这些算法在目标跟踪和预测中有着广泛的应用,但是在处理复杂场景和高速移动目标时,精度和效率存在一定的限制。
\par
基于深度学习方法的算法。
深度学习方法在多目标追踪预测算法中也有广泛的应用\cite{xu2019deep},
比如深度排序追踪器(Deep Learning to Track with Sorting, Deep SORT)\cite{wojke2017simple}等。
这些算法主要利用深度学习网络进行目标检测和特征提取,并使用各种跟踪方法进行目标跟踪和预测。
例如,Deep SORT使用卷积神经网络(CNN)来提取目标特征,并使用传统的匈牙利算法来关联跟踪器。
这些算法的优势在于准确率高、鲁棒性好,但是计算量大,处理速度相对较慢。
\par
融合感知模态的多目标追踪:多模态感知融合是另一个研究热点,旨在利用多个传感器(如摄像头、激光雷达、雷达等)提供的信息来改进多目标追踪性能。通过融合不同传感器的数据,可以提高目标检测和跟踪的鲁棒性,并增强对目标的理解和描述。
\par
端到端的多目标追踪:近年来,出现了一些基于深度学习的端到端多目标追踪方法。这些方法将目标检测、特征提取和目标关联等任务整合到一个网络中,通过联合训练来直接优化多目标追踪的性能。端到端的方法具有简化流程、减少计算复杂性和提高整体性能等优点。

\section{本文的主要研究内容}
本文的研究内容包括以下几点:
\begin{itemize}[itemindent=2em]
    \item 基于RoboMaster比赛环境的工业相机选型。
    \item 基于通信的主从机时钟对齐。
    \item 将传统数字图像处理技术与深度学习相结合的目标检测算法。
    \item 基于卡尔曼滤波器的运动物体建模、预测算法。
    \item 基于物体运动模型的多目标追踪算法。
    \item 受空气阻力的弹道迭代计算方法。
\end{itemize}





