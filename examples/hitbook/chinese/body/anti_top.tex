

\chapter[反陀螺算法设计]{反陀螺算法设计}[Harbin Institute of Technology Postgraduate Dissertation Writing Specifications]

\section{引言}[Content specification]
之所以引入反陀螺算法,是因为在敌方陀螺运动下,
装甲板是圆周运动,具有强烈的非线性,而运动预测只适合线性模型。
且水平速度巨大,反复切换,出现时间短,出现子弹还没有飞过去装甲板就已经消失的情况。


\section{反陀螺模型建立}[Content specification]

基于在笛卡尔坐标系建立的全车观测模型对于观测精度的要求极高,且在中远距离以上(>4m)表现效果不佳。
于是以我方车辆为原点建立圆柱坐标系。原因有二:一是对于陀螺运动的敌方车辆我们更加关注于装甲板在yaw方向的运动;二是相机观测模型保证了yaw方向的观测精度。


在敌方车辆进行陀螺运动时,会发生装甲板切换,在装甲板切换时记录上一次的装甲板和当前装甲板,
记录为小陀螺运动的最左端和最右端。由于车辆四边都有装甲板,则在任意时刻,左右端的中间区域都会有装甲板出现。

反陀螺模型需要计算以下信息:
\begin{itemize}[itemindent=2em]
    \item 前后和左右两组装甲板的平均高度。采用均值滤波算法处理数据。
    \item 整车装甲板的平均高度。采用均值滤波算法处理数据。
    \item 陀螺周期。一次装甲板切换为0.25个周期,采用一阶滞后滤波算法处理数据。
    \item 陀螺方向。通过角速度判断,且经过积分滤波算法处理。
    \item 最左端和最右端分别对应的yaw角度。采用一阶滞后滤波的形式平滑数据。
    \item 最左端和最右端分别对应的yaw角度的等权平均值。采用一阶滞后滤波的形式平滑数据。认为该角度为车辆中心对应的yaw角度。
    \item 装甲板的平均深度信息。采用积分滤波算法处理数据。
\end{itemize}


该算法的核心是计算云台yaw和pitch姿态角。计算过程如下:
我们认为陀螺运动下无论是深度信息的变化(受陀螺运动和平移运动共同影响)还是装甲板高度的变换(受装甲板切换的影响,一般情况下左右和前后两组装甲板的高度不同)
对于子弹飞行时间的影响都是非常小的,即对方在陀螺运动情况下,子弹无论击打到哪个位置的装甲板飞行时间都是相同的。
因此根据上述陀螺模型计算的装甲板的平均深度信息和整车装甲板的平均高度计算子弹飞行时间,
再加入程序延时(当前时刻—上一次装甲板切换时刻),得到一个总体的时间$t$。
根据时间$t$计算陀螺转了整数$n$个1/4圈,且余数为$m$。基于$m$计算yaw角度,计算其相对于最左端或者是最右端(与转动方向有关,如果是往左传,则相对于右端,反之相对于左端)转动的yaw角度。

\begin{lstlisting}
    t_radio = double(4.0 * time / period)- int(4.0 * time / period);
    delta_yaw = yaw_section * t_radio;
\end{lstlisting}

基于$n$计算pitch角度。

\begin{itemize}[itemindent=2em]
    \item 敌方机器人向左转,若$n$是偶数,选择右侧装甲板的高度。
    \item 敌方机器人向左转,若$n$是奇数,选择左侧装甲板的高度。
    \item 敌方机器人向右转,若$n$是偶数,选择左侧装甲板的高度。
    \item 敌方机器人向右转,若$n$是奇数,选择右侧装甲板的高度。
\end{itemize}






\section{高速陀螺模型应对策略}[Content specification]


在装甲板切换时云台需要回调,回调需要时间。如果陀螺转速较低,则云台回调时间相对于云台跟随装甲板的时间可以忽略不计。
随着陀螺转速提高,云台回调时间比重增加,因此需要提前使得云台回调,使得云台回调之后正好迎接下一块装甲板。

\par

如果陀螺转速过高,以至于云台回调时间与云台跟随时间处于同一数量级,
此时将允许弹丸击打到装甲板边缘,具体实现如下:

\begin{lstlisting}
    t_radio = double(4.0 * time / period)- int(4.0 * time / period);
    delta_yaw = yaw_section * t_radio;
\end{lstlisting}

\section{本章小结}[Content specification]

通过分析观测量的精度情况和装甲板的运动情况,在惯性系下建立反陀螺模型的圆柱坐标系,
巧妙的避开了三维坐标测量不准的情况。通过装甲板切换频率判断是否进入反陀螺模型,
基于观测的空间量和时间量计算反陀螺模式的所需信息,实现针对敌方车辆以陀螺模式运动的专门击打算法。
