% !Mode:: "TeX:UTF-8"

\hitsetup{
  %******************************
  % 注意:
  %   1. 配置里面不要出现空行
  %   2. 不需要的配置信息可以删除
  %******************************
  %
  %=====
  % 秘级
  %=====
  statesecrets={公开},
  natclassifiedindex={TM301.2},
  intclassifiedindex={62-5},
  %
  %=========
  % 中文信息
  %=========
  ctitleone={基于卡尔曼滤波器的},%本科生封面使用
  ctitletwo={多目标追踪预测算法},%本科生封面使用
  ctitlecover={基于卡尔曼滤波器的多目标追踪预测算法},%放在封面中使用,自由断行
  ctitle={基于卡尔曼滤波器的多目标追踪预测算法},%放在原创性声明中使用
  % csubtitle={一条副标题}, %一般情况没有,可以注释掉
  cxueke={工学},
  csubject={机器人工程},
  caffil={机电工程学院},
  cauthor={朱纹轩},
  csupervisor={赵明航教授},
  cassosupervisor={某某某教授}, % 副指导老师
  ccosupervisor={某某某教授}, % 联合指导老师
  % 日期自动使用当前时间,若需指定按如下方式修改:
  %cdate={超新星纪元},
  cstudentid={2191300231},
  cstudenttype={学术学位论文}, %非全日制教育申请学位者
  cnumber={no9527}, %编号
  cpositionname={哈铁西站}, %博士后站名称
  cfinishdate={20XX年X月---20XX年X月}, %到站日期
  csubmitdate={20XX年X月}, %出站日期
  cstartdate={3050年9月10日}, %到站日期
  cenddate={3090年10月10日}, %出站日期
  %(同等学力人员)、(工程硕士)、(工商管理硕士)、
  %(高级管理人员工商管理硕士)、(公共管理硕士)、(中职教师)、(高校教师)等
  %
  %
  %=========
  % 英文信息
  %=========
  etitle={Research on key technologies of partial porous externally pressurized gas bearing},
  esubtitle={This is the sub title},
  exueke={Engineering},
  esubject={Computer Science and Technology},
  eaffil={\emultiline[t]{School of Mechatronics Engineering \\ Mechatronics Engineering}},
  eauthor={Yu Dongmei},
  esupervisor={Professor XXX},
  eassosupervisor={XXX},
  % 日期自动生成,若需指定按如下方式修改:
  edate={December, 2017},
  estudenttype={Master of Art},
  %
  % 关键词用“英文逗号”分割
  ckeywords={全局快门, 卷积神经网络, 卡尔曼滤波, 多目标追踪},
  ekeywords={\TeX, \LaTeX, CJK, template, thesis},
}

\begin{cabstract}
  本文提出了一种基于卡尔曼滤波器的多目标追踪预测算法;
  在相机选型方面,选择了全局快门方式、高帧率的相机以提高检测动态目标的能力;
  在主从机时间帧对齐方面,基于通信设计了动态时间帧对齐,保证了主从机使用同一时钟源;
  在目标检测算法方面,采用了将传统数字图像处理与深度学习相结合的方案,
  能够高效准确地检测到复杂光照环境下的目标;
  在运动预测算法方面,
  使用了基于卡尔曼滤波器的运动模型,对目标的位置和速度进行预测;
  在追踪算法方面,使用了类贪心算法的多目标追踪算法,
  提高追踪的准确性和效率。实验结果表明,本算法能够高效准确地完成多目标追踪预测任务,并且具有较高的实时性和鲁棒性。
\end{cabstract}

\begin{eabstract}
   An abstract of a dissertation is a summary and extraction of research work
   and contributions. Included in an abstract should be description of research
   topic and research objective, brief introduction to methodology and research
   process, and summarization of conclusion and contributions of the
   research. An abstract should be characterized by independence and clarity and
   carry identical information with the dissertation. It should be such that the
   general idea and major contributions of the dissertation are conveyed without
   reading the dissertation.

   An abstract should be concise and to the point. It is a misunderstanding to
   make an abstract an outline of the dissertation and words ``the first
   chapter'', ``the second chapter'' and the like should be avoided in the
   abstract.

   Key words are terms used in a dissertation for indexing, reflecting core
   information of the dissertation. An abstract may contain a maximum of 5 key
   words, with semi-colons used in between to separate one another.
\end{eabstract}
