% !Mode:: "TeX:UTF-8" 



\begin{conclusions}

本文介绍了一种基于卡尔曼滤波器的多目标追踪预测算法。
该算法的目的是通过结合高帧率的全局快门相机、基于通信的时间同步协议、
基于传统图像处理算法与深度学习技术结合的目标检测算法、
基于卡尔曼滤波器的运动预测算法
和基于贪心算法的多目标追踪算法等技术手段,实现高效准确的多目标追踪预测任务。

在本文中,我们通过实验验证了该算法的可行性和有效性。实验结果表明,该算法能够在多目标场景下实现高效准确的追踪预测,并且具有较高的实时性和鲁棒性。以下是本文的结论总结:


高帧率的全局快门相机有助于提高多目标追踪预测的准确性。本文采用了高帧率的全局快门相机,
提高了图像的采集频率,从而为目标检测算法提供了更精确的输入。

基于通信的动态时间帧对齐可以保证主从机使用同一时钟源,不仅可以使得识别目标的坐标转换更加准确,
也有利于控制系统在收到滞后控制信息的时候根据时间差补偿,提高了对于动态物体的击打命中率。

基于传统图像处理算法与深度学习技术结合的目标检测算法,
能够高效准确地检测到各种形态的目标。

基于卡尔曼滤波器的运动预测算法可以提高多目标追踪预测的准确性。
本文采用了基于卡尔曼滤波器的运动预测算法,对目标的位置和速度进行预测,从而提高了多目标追踪预测的准确性。

基于贪心算法的多目标追踪算法能够在短时间内得出可接受的解决方案。
本文采用了基于贪心算法的多目标追踪算法,能够在短时间内得出可接受的多目标匹配方案,提高了多目标追踪预测的效率。
综上所述,本文提出的基于卡尔曼滤波器的多目标追踪预测算法,通过结合多种技术手段,实现了高效准确的多
目标追踪预测任务。
该算法具有实现简单、运算速度快、准确度高、鲁棒性强等优点。

然而,该算法仍然存在一些不足之处。
例如,对于极端光照环境、装甲板受遮挡、远距离等情况下,目标检测算法的准确性会受到影响;
对于超机动目标,受限于匀速运动模型的影响,卡尔曼滤波器的预测精度也会受到影响。
因此,未来可以进一步研究如何结合更加先进的目标检测算法
(如端到端直接回归装甲板端点)和更适合机动目标的运动模型(如Singer\cite{2013A}模型),以提高算法的鲁棒性和精度。



\end{conclusions}
