\section{论文工作是否按预期进行、目前已完成的研究工作及结果}
\subsection{论文工作是否按预期进行}
论文工作按照预期进行: \par
截止2022.12月份:代码框架搭建完成, 基于相机的SDK开发相机取图进程,实现检测部分的传统算法。\par
截止2022.2月份:实现检测部分的深度学习部分:数据集制作、网络训练、网络部署。\par
截止2022.3月份:实现基于卡尔曼滤波器的追踪和预测部分。\par
\subsection{目前已完成的研究工作及结果}
\section{后期拟完成的研究工作及进度安排}
\subsection{后期拟完成的研究工作}
1. 完成单目相机测距算法。
2. 完成受空气阻力的弹道迭代计算。
3. 充分测试系统的鲁棒性,如识别算法在不同光照环境下的表现效果,预测算法在角度跨圈时处理是否得当,滤波算法是否能够很好的应对不同的噪声。
4. 与控制系统联合调试,测试实际效果。
\subsection{后期进度安排}
\section{存在的问题与困难}
1. 赛场环境非常复杂,而图像预处理的二值化操作严重依赖于阈值的选择。相同曝光时间下,因环境亮度不同, 二值化阈值也不同,
需要设计硬件自动曝光与软件自动曝光算法,使得在不同光照环境下得到的图像亮度基本保持一致。\par

2. 数字识别在低亮度环境下表现效果不好。因为要追求高帧率的目标检测,假设目标帧率为 150fps,
则最大曝光时间不超过6.7ms,在弱光照环境下成像较暗,卷积分类网络表现效果不好。针对此问题,计划通过两种方式解决:
- 增加在此光照环境下的数据集,让神经网络多学习这一场景。
- 由于数字图案的背景都是黑色的,而数字本身是白色的,通过大津二值化算法实现数字与背景的图像分割,让数字图案更加清晰,数字特征更加明显。\par
\section{论文按时完成的可能性}
论文能够保证按时完成,主要技术问题都得到解决,剩下的只是一些小的细节和优化。\par
\section{参考文献}
\bibliographystyle{hithesis}
\bibliography{reference}

% Local Variables:
% TeX-master: "../mainart"
% TeX-engine: xetex
% End: