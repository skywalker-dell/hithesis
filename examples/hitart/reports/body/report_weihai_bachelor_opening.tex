% !Mode:: "TeX:UTF-8"
\section{课题背景及研究的目的和意义}
\subsection{课题背景}
本项目基于全国大学生机器人大赛RoboMaster超级对抗赛,赛事规则充分融合了“机器视觉”、“嵌入式系统设计”、“机械控制”、“惯性导航”、“人机交互”等众多机器人相关技术学科。
在该比赛中,操作手需要驾驶自己研发的车辆,通过发射17mm和42mm弹丸实现对装甲板的击打,基地剩余血量较高者获胜。
通过视觉辅助提高对运动装甲板的命中率是获得比赛的关键之一。

\subsection{研究的目的和意义}
本研究旨在为战队研发一种高效鲁棒的自动瞄准算法,实现对装甲板的识别、追踪、坐标解算以及运动预测等算法。一键瞄准、自动锁定,帮助操作手提高装甲板的命中率,从而取得比赛的胜利。
同时,该研究涉及大量的机器人领域的知识,并涉及实际的工程开发,学以致用,提高学生对专业的掌握。
\section{国内外在该方向的研究现状及分析}
\subsection{国外现状及分析}
目标检测是计算机视觉中最重要的实际应用之一。总体上的研究思路可以分为传统算法和深度学习两种。
传统算法做主要分为背景建模法与前景建模法,主要的研究思路是:区域选取、特征提取、特征分类。
深度学习则可以分为一阶段检测算法和两阶段检测算法。
二阶段检测算法以R-CNN、Fast R-CNN、 Faster R-CNN为代表,这类算法精度较高,但是其推理速度较慢;
一阶段检测算法以YOLO为代表,兼顾精度与速度,是目前常用的检测算法。 \par

在这里我们仅讨论基于可见光图像的多目标追踪(MOT)。
在多目标追踪领域,SORT是一种非常具有代表性的算法。其原理是通过Faster R-CNN检测目标,得到检验框后通过
线性卡尔曼滤波预测它们在下一帧中的位置,然后将位置预测结果与目标检测框通过IOU值进行匈牙利匹配,从而获得追踪框。
之后又有一些改进的算法,如:Deep SORT等。



\subsection{国内现状及分析}
国内研究现状与国外大致类似,值得注意的是在深度学习领域,中国学者的研究紧跟世界前沿。
\section{研究内容及拟解决的关键问题}
\subsection{研究内容}
1.主从机通信。在NUC(上位机)部署视觉算法等,在STM32(下位机)部署控制器算法,NUC发送云台期望角度信息给下位机,下位机反馈当前云台状态(角度等)。\par
2.鲁棒的目标检测算法,能够适应各种光照环境,且算法轻量化,为了控制器能够平滑跟随且不产生明显滞后,将算法部署在计算平台NUC11上,期望目标帧率150fps+。 \par
3.准确的状态估计,单目相机采用PnP算法测距,深度噪声大,通过对原始数据滤波得到较为平滑的数据,在此基础上拟合其运动状态。 \par
4.坐标系转换,设计主从机对时系统,得到带有时间戳的陀螺仪信息,方便坐标系转换。 \par
5.多线程、多进程编程。将相机取图、视觉算法、状态估计模块、通信模块解耦,最大化使用CPU多核性能,提高算法运行速度,且使得整体框架清晰,易于维护。\par
6.受空气阻力的弹道解算与预瞄点的迭代计算。\par
7.通过系统辨识设计云台控制器。\par
\subsection{拟解决的关键问题}
1.目标检测算法能够适应各种复杂的光照环境。\par
2.高帧率的目标检测算法,设计帧率高于150帧。 \par
3.精确的单目测距算法,期望误差小于$10\%$。 \par
4.高帧率、大分辨率的图像流的传输方式一定程度上决定着总体算法的处理帧率,如何高效率在不同线程、进程间传递图像数据。 \par
5.单目相机的测距误差较大,如何对数据平滑又不造成明显的滞后需要大量实验调参 \par
6.不同信息源的数据的时间同步
\section{拟采取的研究方法和技术路线、进度安排、预期达到的目标}
\subsection{拟采取的研究方法和技术路线}
1.设计自动曝光算法,根据当前光照环境自动调节曝光值。
2.采用传统数字图像处理算法提取轮廓特征,并采用卷积神经网络去除误分类的方式
3.采用ROS提供的message_filter工具实现不同信息源数据的时间同步
4.采用PnP算法实现单目相机测距
5.
\subsection{进度安排}
\subsection{预期达到的目标}
\section{课题已具备和所需的条件}
1.硬件设备已具备。 
2.
\section{研究过程中可能遇到的困难和问题,解决的措施}

\section{参考文献}
\bibliographystyle{hithesis}
\bibliography{reference}

% Local Variables:
% TeX-master: "../report"
% TeX-engine: xetex
% End: