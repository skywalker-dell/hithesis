% !Mode:: "TeX:UTF-8"
\section{课题背景及研究的目的和意义}
\subsection{课题背景}
本项目基于全国大学生机器人大赛RoboMaster超级对抗赛,赛事规则充分融合了“机器视觉”、“嵌入式系统设计”、“机械控制”、“惯性导航”、“人机交互”等众多机器人相关技术学科。
在该比赛中,操作手需要驾驶自己研发的车辆,通过发射17mm和42mm弹丸实现对装甲板的击打,基地剩余血量较高者获胜。
通过视觉辅助提高对运动装甲板的命中率是获得比赛的关键之一。

\subsection{研究的目的和意义}
本研究旨在为战队研发一种高效鲁棒的自动瞄准算法,实现对装甲板的识别、追踪、坐标解算以及运动预测等算法。一键瞄准、自动锁定,帮助操作手提高装甲板的命中率,从而取得比赛的胜利。
同时,该研究涉及大量的机器人领域的知识,并涉及实际的工程开发,学以致用,提高学生对专业的掌握。
\section{国内外在该方向的研究现状及分析}
\subsection{国外现状及分析}
目标检测是计算机视觉中最重要的实际应用之一。总体上的研究思路可以分为传统算法和深度学习两种。
传统算法做主要分为背景建模法与前景建模法,主要的研究思路是:区域选取、特征提取、特征分类。
深度学习则可以分为一阶段检测算法和两阶段检测算法。
二阶段检测算法以R-CNN、Fast R-CNN、Faster R-CNN为代表,这类算法精度较高,但是其推理速度较慢;
一阶段检测算法以YOLO为代表,兼顾精度与速度,是目前常用的检测算法。 \par

关于目标追踪我们仅讨论基于可见光图像的多目标追踪MOT。
在多目标追踪领域,SORT是一种非常具有代表性的算法。其原理是通过Faster R-CNN检测目标,得到检验框后通过
线性卡尔曼滤波预测它们在下一帧中的位置,然后将位置预测结果与目标检测框通过IOU值进行匈牙利匹配,从而获得追踪框。
之后又有一些改进的算法,如:Deep SOR等。

从20世纪80年代甚至更早,高斯模型、线性近似模型和位置跟踪组成了最初的Kalman Filters
框架,在95年代,离散的方法论开始在拓扑型表征和栅格模型表征上进行全局定位、重定位和不确定处理(POMDPs)的研究,
再到了99年,无参滤波器Particle Filters开始在基于采样方法的定位、重定位方向上大展身手,
现在,各个假说和前提条件下的滤波器研究成了研究的热门,如EKF的各个变体:ESKF、MSCKF和IKF都是
针对不同的应用情况和追求被提出并得到了广泛的应用和优化。

ROS是一个用于编写机器人软件的框架,它提供了许多工具,同时也制定了一系列的标准。
目的就是通过标准来规范软件开发,通过接口来提高模块的复用率,通过提供一系列工具方便开发和调试。
目前,社区已经推出ROS2。其特点是:
广泛使用C++11、跨平台、采用DDS替代原有的UDP通信机制增强实行性、更稳定的编译系统、
更好的支持多机器人系统等。


\subsection{国内现状及分析}
国内研究现状与国外大致类似,值得注意的是在深度学习领域,中国学者的研究紧跟世界前沿。
\section{研究内容及拟解决的关键问题}
\subsection{研究内容}
1.主从机通信。在NUC(上位机)部署视觉算法等,在STM32(下位机)部署控制器算法,NUC发送云台期望角度信息给下位机,下位机反馈当前云台状态(角度等)。\par
2.鲁棒的目标检测算法,能够适应各种光照环境,且算法轻量化,为了控制器能够平滑跟随且不产生明显滞后,将算法部署在计算平台NUC11上,期望目标帧率150fps+。 \par
3.准确的状态估计,单目相机采用PnP算法测距,深度噪声大,通过对原始数据滤波得到较为平滑的数据,在此基础上拟合其运动状态。 \par
4.坐标系转换,设计主从机对时系统,得到带有时间戳的陀螺仪信息,方便坐标系转换。 \par
5.多线程、多进程编程。将相机取图、视觉算法、状态估计模块、通信模块解耦,最大化使用CPU多核性能,提高算法运行速度,且使得整体框架清晰,易于维护。\par
6.受空气阻力的弹道解算与预瞄点的迭代计算。\par
7.通过系统辨识设计云台控制器。\par
\subsection{拟解决的关键问题}
1.目标检测算法能够适应各种复杂的光照环境。\par
2.高帧率的目标检测算法,设计帧率高于150帧。 \par
3.精确的单目测距算法,期望误差小于$10\%$。 \par
4.高帧率、大分辨率的图像流的传输方式一定程度上决定着总体算法的处理帧率,如何高效率在不同线程、进程间传递图像数据时提升算法整体帧率的关键之一。 \par
5.单目相机的测距误差较大,如何对数据平滑又不造成明显的滞后需要大量实验调参。 \par
6.不同信息源的数据的时间同步。\par
\section{拟采取的研究方法和技术路线、进度安排、预期达到的目标}
\subsection{拟采取的研究方法和技术路线}
1.设计自动曝光算法,根据当前光照环境自动调节曝光值。 \par
2.采用传统数字图像处理算法提取轮廓特征,并采用卷积神经网络去除误分类的方式。 \par
3.采用ROS提供的message\_filter工具实现不同信息源数据的时间同步。 \par
4.采用PnP算法实现单目相机测距。 \par
5.采用卡尔曼滤波器实现数据平滑。 \par
\subsection{进度安排}
1. 当前 - 2022.12: 代码框架搭建完成, 基于相机的SDK开发相机去图进程,实现检测部分的传统算法。 \par
2. 2023.01 - 2023.02: 实现检测部分的深度学习部分:数据集制作、网络训练、网络部署。\par
3. 2023.02 - 2023.03: 实现基于卡尔曼滤波器的追踪和预测部分。\par
4. 2023.03 - 2023.04: 完成代码剩余部分(如弹道解算部分等)并上机调试。\par
5. 2023.04 - 中期答辩: 根据算法效果改进相关代码,尽量提升性能。\par
\subsection{预期达到的目标}
云台能够流畅的跟踪目标,发射出的子弹击打速度范围在0-4/ms的装甲板命中率在$90\%$以上的命中率。
\section{课题已具备和所需的条件}
已具备条件:工业相机、移动计算设备。 \par
所需条件: GPU设备(用于训练神经网络)、装甲板数据集。 \par
\section{研究过程中可能遇到的困难和问题,解决的措施}
1. 制作数据集需要耗费大量的时间和精力,需要在不同位置、不同光照环境下的拍摄的图片。 \par
2. 由于无法评估所需网络结构的复杂度 目前无法评估神经网络对于算力的影响. \par
\newpage
\section{参考文献}
\bibliographystyle{hithesis}
\bibliography{reference}

% Local Variables:
% TeX-master: "../report"
% TeX-engine: xetex
% End: